\documentclass[12pt]{report}

\usepackage[dutch]{babel}
\usepackage[nobottomtitles]{titlesec}
\usepackage[bottom]{footmisc}
\usepackage{graphicx}
\usepackage{titleps}
\usepackage{amssymb}
\usepackage{amsmath}
\usepackage{amsthm}
\usepackage{graphicx}
\usepackage{verbatim}
\usepackage{titling}
\usepackage[toc,page]{appendix}
\usepackage{bm}

\usepackage{fancyhdr}

\pagestyle{fancy}

\renewcommand{\chaptermark}[1]{\markboth{#1}{}}
\renewcommand{\sectionmark}[1]{\markright{#1}}
\fancyhf{}
\fancypagestyle{plain}{%
  \fancyhf{}%
	\rfoot{\fancyplain{}{\nouppercase{\thepage}}}
	\lfoot{\fancyplain{}{Thorvald Dox}}
  \renewcommand{\headrulewidth}{0pt}% Line at the header invisible
  \renewcommand{\footrulewidth}{0.4pt}% Line at the footer visible
}
\lhead{\fancyplain{}{Semilineaire MESMA}}
\rhead{\fancyplain{}{\nouppercase{\leftmark}}}
\rfoot{\fancyplain{}{\nouppercase{\thepage}}}
\lfoot{\fancyplain{}{Thorvald Dox}}
\renewcommand{\footrulewidth}{0.4pt}

\fancyhfoffset[E,O]{0pt}
\date{}
  
\usepackage[margin=1in]{geometry}
\usepackage{float}



\usepackage[super,sort]{natbib}
\usepackage{bibentry}
\nobibliography*

\newcommand{\footcite}[1]{\cite{#1}\let\thefootnote\relax \footnote{\cite{#1} \bibentry{#1}} }

\DeclareMathOperator*{\Odot}{\bigodot}
\allowdisplaybreaks

%\title{Semi-lineair Multiple Endmember mixture spectrum analysis}
\title{Semilineaire spectraalanalyse met meerdere eindmembers}
\author{Thorvald Dox}

\begin{document}
\begin{titlepage}

\newcommand{\HRule}{\rule{\linewidth}{0.5mm}} % Defines a new command for the horizontal lines, change thickness here

\center % Center everything on the page
 
%----------------------------------------------------------------------------------------
%	HEADING SECTIONS
%----------------------------------------------------------------------------------------

\textsc{\LARGE Universiteit Antwerpen}\\[1.5cm] % Name of your university/college
\textsc{\Large Fysica}\\[0.5cm] % Major heading such as course name
\textsc{\large Masterproef}\\[0.5cm] % Minor heading such as course title

%----------------------------------------------------------------------------------------
%	TITLE SECTION
%----------------------------------------------------------------------------------------

\HRule \\[0.4cm]
{ \Large \bfseries \thetitle}\\[0.4cm] % Title of your document
\HRule \\[1.5cm]
 
%----------------------------------------------------------------------------------------
%	AUTHOR SECTION
%----------------------------------------------------------------------------------------

\begin{minipage}{0.4\textwidth}
\begin{flushleft} \large
\emph{Auteur:}\\
\theauthor % Your name
~\\
~\\
~\\
\end{flushleft}
\end{minipage}
~
\begin{minipage}{0.4\textwidth}
\begin{flushright} \large
\emph{Promotor:} \\
Paul {Scheunders} \\ % Supervisor's Name
\emph{Copromotor:} \\
Rob {Heylen} % Supervisor's Name
\end{flushright}
\end{minipage}\\[5cm]

% If you don't want a supervisor, uncomment the two lines below and remove the section above
%\Large \emph{Author:}\\
%John \textsc{Smith}\\[3cm] % Your name

%----------------------------------------------------------------------------------------
%	DATE SECTION
%----------------------------------------------------------------------------------------

%{\large \today}\\[3cm] % Date, change the \today to a set date if you want to be precise

%----------------------------------------------------------------------------------------
%	LOGO SECTION
%----------------------------------------------------------------------------------------

%\includegraphics[height=4cm]{remote.png}
\includegraphics[height=4cm]{download.jpg}
\hspace{5 cm}
\includegraphics[height=4cm]{vlabsym.png}
 \\[1cm] % Include a department/university logo - this will require the graphicx package

 
%----------------------------------------------------------------------------------------

\vfill % Fill the rest of the page with whitespace

\end{titlepage}
\tableofcontents
\newpage
\chapter*{Abstract}
\addcontentsline{toc}{chapter}{Abstract}

\newpage
\chapter*{Inleiding}
\addcontentsline{toc}{chapter}{Inleiding}
\section{aardopservatie}



\chapter{ontmengen}

\section{spectrale analyse}

\begin{itemize}
\item spectra als functies (eigenschappen van licht)
\item spectra als vector (endmembers) $\rightarrow$ matlab implementatie
\item mengen van endmembers (abundancies)
\end{itemize}

\subsection{atmosferische correctie}

\section{ontmengen}

\begin{itemize}
\item optimalisatieprobleem
\item reconstructie
\item reconstructieerror
\item vrijheidsgraden
\end{itemize}

\section{lineair ontmengen}

Least-Squares
$\rightarrow$ Berekening


\subsubsection{niet-negativiteit}

\subsubsection{sum to one}



\subsection{implementatie in matlab}

\section{multilineair ontmengen}

Uitleg $\rightarrow$ lichtstraal heeft kans om te reflecteren

\subsection{berekening}

\subsection{reflectancy vs albedo}

\subsection{afhankelijke vs onafhankelijke P waarden}

\subsection{Ondergrens van P waarde}

Of dat $P > 0$ moet gebruikt worden als voorwaarde of niet.

\subsection{implementatie in matlab}

\chapter{Selecteren}

\section{Variabiliteit}

\begin{itemize}
\item variabiliteit
\item bibliotheek $\rightarrow$ model
\item pixel-afhankelijke selectie
\end{itemize}

\section{MESMA}

\begin{itemize}
\item Ontmengen aan de hand van elke subset
\item Selectie op basis van niet-negativiteitsvoorwaarde
\end{itemize}

\subsection{ontmengingsmethode naar keuz (lineair vs multilineair)}

\subsection{implementatie in matlab}

\section{AAM}

\begin{itemize}
\item concept hoek in hoogdimentonale ruimtes
\item AAM
\end{itemize}

\subsection{implementatie in matlab}


\chapter{Methodes}

\section{Semi-lineair model}

ontkoppeling van Ontmenging in MESMA bij selectie tov ontmenging voor abundancies

\subsection{Theoretische controle dmv monte carlo simulaties}

\section{multilineair AAM}

\chapter{experimentele vergelijking van verschillende methodes}

\section{Alina dataset}

\section{Lijst en korte uitleg bij alle methodes}

Elke methode is hiervoor beschreven, maar dit beschrijft kort de verschillen in de methodes en hoe deze geimplementeerd zijn door middel van `schakelaars' in de code. Ook een uitleg bij de weergave van de resultaten. 

\begin{itemize}
\item lineair MESMA
\item semi-lineair MESMA
\item multi-lineair MESMA
\item lineair AAM
\item multilineair AAM
\end{itemize}

$\rightarrow$ Voor de multilineaire modellen wordt ook al dan niet $P> 0$ en $P$ materiaalafhankelijk vergeleken.



\section{Bepreking lineair vs semi-linair}

Verschil voor hoge reflectie (bomen)
$\rightarrow$ semilinair geeft betere resultaten voor dezelfde runtime

\section{Bespreking semi-lineair vs multilineair}

$\rightarrow$ semilineair geeft vergelijkbare resultaten op kortere tijd

\section{Bespreking P-afhankelijkheid}

$\rightarrow$ P-afhankelijkheid geeft vergelijkbare resultaten op gelijke tijd, maar heeft meer vrijheidsgraden

\section{Bespreking P-ondergrens}

verschil voor lage reflectie (asfalt)
$\rightarrow$ Weglaten van ondergrens geeft betere resultaten op gelijke tijd.

\section{Bepreking multilinair AAM vs semilinair model}

$\rightarrow$ zelfde resultaten voor kortere tijd.

\begin{appendices}


\end{appendices}


\begin{flushleft}
\nocite{*}
\bibliography{biblio}{}
\bibliographystyle{amsplain}
\addcontentsline{toc}{chapter}{Bibliografie}

\end{flushleft}


\end{document}
